\documentclass[a5paper,10pt]{article}
\usepackage[utf8]{inputenc}
\usepackage[T1]{fontenc}
\usepackage{amsmath}
\usepackage{amsfonts}
\usepackage{amssymb}
\usepackage{xcolor}
\usepackage{geometry}
\usepackage{ragged2e}
\usepackage{environ}
\usepackage{framed}
\usepackage{graphicx}


\geometry{a5paper, margin=2cm, top=2.5cm, bottom=2.5cm, headheight=1.5cm, footskip=1cm}


\definecolor{formulaColor}{RGB}{20,50,150}
\definecolor{legendColor}{RGB}{60,60,60}
\definecolor{formulabgcolor}{gray}{0.95}

\newenvironment{displayformula}
{
	\begin{framed}
		\color{formulaColor}
	}
	{\end{framed}}

\newcommand{\formulalegend}[1]{%
	\par\vspace{0.5ex}%
	{{\color{legendColor}\RaggedRight\small\textit{#1}}}%
	\par\vspace{1.5ex}%
}

\setcounter{tocdepth}{3}

\begin{document}
	\tableofcontents
	\newpage
	
	\section{Physikalische Größen und Einheiten}
	
	\subsection{Messunsicherheit Typ A}
	
	\begin{displayformula}
		Arithmetischer Mittelwert
		\[
		\bar{x} = \frac{1}{N} \sum_{i=1}^{N} x_i
		\]
	\end{displayformula}
	\formulalegend{
		\( \bar{x} \): Mittelwert der Messwerte [Einheit wie \( x_i \)], \( x_i \): Einzelne Messwerte, \( N \): Anzahl der Messungen
	}
	
	\begin{displayformula}
		Standartabweichung eines Messwertes
		\[
		\Delta x = \frac{1}{N - 1} \sum_{i = 1}^{N} (x_i - \bar{x})^2
		\]	
	\end{displayformula}
	\formulalegend{
		\( \Delta x \): Standardabweichung [Einheit wie \( x_i \)], \( x_i \): Einzelne Messwerte, \( \bar{x} \): Mittelwert, \( N \): Anzahl der Messwerte
	}
	
	\begin{displayformula}
		Standartabweichung des Mittelwertes
		\[
		\Delta \bar{x} = \frac{\Delta x}{\sqrt{N}} 
		\]	
	\end{displayformula}
	\formulalegend{
		\( \Delta \bar{x} \): Standardabweichung des Mittelwertes, \( \Delta x \): Standardabweichung [Einheit wie \( x_i \)], \( N \): Anzahl der Messwerte
	}
	
	\begin{displayformula}
		Darstellung der Messgröße x
		\[
		x_p = \bar{x} \pm t_p \cdot \Delta x
		\]
		\[
		x_p = \bar{x} \pm U_a (x)
		\]
	\end{displayformula}
	\formulalegend{
		\( x_p \): Messgröße, \( \bar{x} \): Mittelwert, \( t_p \): Vertrauensfaktor, \( \Delta x \): Standardabweichung, \( U_a(x) \): erweiterte Unsicherheit
	}
	
	\subsection{Messunsicherheit Typ B}
	\begin{displayformula}
		Unsicherheiten, welche nicht durch Wiederholungsmessungen ermittelt werden. \\ Die Messunsicherheit ist angegeben
	\end{displayformula}
	
	\subsubsection{Ermittlung des kombinierten Unsicherheit}
	
	\begin{displayformula}
		Wenn Typ A und Typ B vorliegen
		\[
		U_\text{mess} = \sqrt{U^2_A(x) + U^2_{B_1}(x) + U^2_{B_2}(x) + \dots}
		\]
	\end{displayformula}
	\formulalegend{
		\( U_\text{mess} \): Gesamte kombinierte Unsicherheit, \( U_A(x) \): Unsicherheit Typ A, \( U_{B_i}(x) \): Unsicherheit Typ B
	}
	
	\newpage
	
\section{Verschiebung, Geschwindigkeit und \\Geschwindigkeitsbetrag}

\begin{displayformula}
	Verschiebung
	\[
	\Delta x = x_E - x_A
	\]
\end{displayformula}
\formulalegend{
	\( \Delta x \): Verschiebung [m], \( x_E \): Endposition [m], \( x_A \): Anfangsposition [m]
}

\begin{displayformula}
	Mittlere Geschwindigkeit
	\[
	\bar{v}_x = \frac{\Delta x}{\Delta t}
	\]
\end{displayformula}
\formulalegend{
	\( \bar{v_x} \): Mittlere Geschwindigkeit [m/s], \( \Delta x \): Weg [m], \( \Delta t \): Zeitintervall [s]
}

\begin{displayformula}
	Momentangeschwindigkeit
	\[
	v_x = \dfrac{x}{t} = \dot{x} (t)
	\]
\end{displayformula}
\formulalegend{
	\( v_x \): Momentangeschwindigkeit [m/s], \( x \): Position [m], \( t \): Zeit [s], \( \dot{x}(t) \): Ableitung von \( x(t) \) nach der Zeit
}

\newpage

\section{Gleichförmig beschleunigte Bewegung}

\begin{displayformula}
	Der mittlere Geschwindigkeitsbetrag (speed) \( \bar{v}_x \) ist definiert als zurückgelegte Strecke \( s \) geteilt durch die benötigte Zeit \( \Delta t \):
	\[
	\bar{v}_x = \frac{s}{\Delta t}
	\]
\end{displayformula}
\formulalegend{
	\( \bar{v}_x \): Mittlere Geschwindigkeit [m/s], \( s \): Strecke [m], \( \Delta t \): Zeitintervall [s]
}

\begin{displayformula}
	Die mittlere Beschleunigung \( \bar{a}_x \) ist definiert als Änderung der Geschwindigkeit \( v_x \) pro Zeiteinheit \( \Delta t \):
	\[
	\bar{a}_x = \frac{v_{xE} - v_{xA}}{\Delta t}
	\]
\end{displayformula}
\formulalegend{
	\( \bar{a}_x \): Mittlere Beschleunigung [m/s²], \( v_{xE} \): Endgeschwindigkeit [m/s], \( v_{xA} \): Anfangsgeschwindigkeit [m/s], \( \Delta t \): Zeit [s]
}

\newpage

\section{Gleichmäßig beschleunigte Bewegung}

\begin{displayformula}
	\[
	x(t) = x_0 + v_{x0}t + \frac{1}{2} a_x t^2 
	\]
	\[
	v_x(t) = v_{x0} + a_x t
	\]
	\[
	a_x(t) = a_x
	\]
\end{displayformula}
\formulalegend{
	\( x(t) \): Position zur Zeit \( t \) [m], \( x_0 \): Anfangsposition [m], \( v_{x0} \): Anfangsgeschwindigkeit [m/s], \( a_x \): konstante Beschleunigung [m/s²], \( t \): Zeit [s]
}

\section{Bewegung in zwei und drei Dimensionen}

\begin{displayformula}
	\[
	r(t) = x(t)\vec{e_x} + y(t)\vec{e_y}
	\]
	\[
	= \begin{pmatrix}
		x(t) \\
		y(t)
	\end{pmatrix}
	\]
\end{displayformula}
\formulalegend{
	\( r(t) \): Ortsvektor [m], \( x(t), y(t) \): Komponenten der Position [m], \( \vec{e_x}, \vec{e_y} \): Einheitsvektoren
}

\begin{displayformula}
	\[
	\Delta \vec{r}(t) = \vec{r_E}(t) - \vec{r_A}(t)
	\]
	\[
	= \begin{pmatrix}
		x_E(t) - x_A(t) \\
		y_E(t) - y_A(t)
	\end{pmatrix}
	\]
\end{displayformula}
\formulalegend{
	\( \Delta \vec{r}(t) \): Verschiebungsvektor [m], \( \vec{r_E}(t) \): Endposition, \( \vec{r_A}(t) \): Anfangsposition
}

\begin{displayformula}
	Mittlere Geschwindigkeit
	\[
	\vec{v} = \frac{\Delta \vec{r}}{\Delta t}
	\]
\end{displayformula}
\formulalegend{
	\( \vec{v} \): Mittlere Geschwindigkeit [m/s], \( \Delta \vec{r} \): Verschiebung [m], \( \Delta t \): Zeitintervall [s]
}

\begin{displayformula}
	\[
	\vec{r}(t) = x(t)\vec{e_x} + y(t)\vec{e_y} + z(t)\vec{e_z}
	\]
	\[
	=
	\begin{pmatrix}
		x(t)\\
		y(t) \\
		z(t)
	\end{pmatrix}
	\]
	z.B.
	\[
	\vec{r}(t) = \begin{pmatrix}
		x_0 + v_{x0}t + \frac{1}{2}a_x t^2 \\
		y_0 + v_{y0}t + \frac{1}{2}a_y t^2 \\
		z_0 + v_{z0}t + \frac{1}{2}a_z t^2 
	\end{pmatrix}
	\]
\end{displayformula}
\formulalegend{
	\( \vec{r}(t) \): Ortsvektor [m], \( x_0, y_0, z_0 \): Anfangskoordinaten [m], \( v_{x0}, v_{y0}, v_{z0} \): Anfangsgeschwindigkeiten [m/s], \( a_x, a_y, a_z \): Beschleunigungen [m/s²], \( t \): Zeit [s]
}
\newpage

\subsection{Der schräge Wurf}

\begin{displayformula}
	\[
	\vec{r}(t) =
	\begin{pmatrix}
		v_0 \cdot \cos\alpha \cdot t\\
		y_0 + v_0\sin\alpha \cdot t - \frac{1}{2} \cdot g \cdot t^2
	\end{pmatrix}
	=
	\begin{pmatrix}
		v_{x0} t\\
		y_0 + v_{y0} t - \frac{1}{2} g t^2
	\end{pmatrix}
	\]
\end{displayformula}
\formulalegend{
	\( \vec{r}(t) \): Ortsvektor [m], \( v_0 \): Anfangsgeschwindigkeit [m/s], \( \alpha \): Abwurfwinkel, \( g \): Erdbeschleunigung [m/s²], \( t \): Zeit [s], \( y_0 \): Anfangshöhe [m]
}

\begin{displayformula}
	\[
	y(t) = y_0 + v_0\sin\alpha \cdot t - \frac{1}{2} gt^2
	\]
\end{displayformula}
\formulalegend{
	\( y(t) \): Höhe zur Zeit \( t \) [m], \( y_0 \): Anfangshöhe [m], \( v_0 \): Anfangsgeschwindigkeit [m/s], \( \alpha \): Winkel, \( g \): Erdbeschleunigung [m/s²], \( t \): Zeit [s]
}

\begin{displayformula}
	\[
	y(x) = y_0 + v_0\sin\alpha \cdot \frac{x}{v_0 \cos\alpha} - \frac{1}{2} g \left(\frac{x}{v_0\cos\alpha}\right)^2
	\]
\end{displayformula}
\formulalegend{
	\( y(x) \): Höhe in Abhängigkeit vom horizontalen Ort \( x \) [m], \( y_0 \): Anfangshöhe [m], \( v_0 \): Anfangsgeschwindigkeit [m/s], \( \alpha \): Winkel, \( g \): Erdbeschleunigung [m/s²], \( x \): horizontale Entfernung [m]
}
\newpage


\section{Die Newtonschen Axiome}

\begin{displayformula}
	\[
	F = m \cdot a
	\]
\end{displayformula}
\formulalegend{
	\( F \): Kraft [N], \( m \): Masse [kg], \( a \): Beschleunigung [m/s²]
}

\subsection{Das erste Newtonsche Axiom: Das Trägheitsgesetz}

\begin{displayformula}
	\[
	\vec{a} = 0 \quad \text{falls} \quad \vec{F} = 0
	\]
\end{displayformula}
\formulalegend{
	\( \vec{a} \): Beschleunigung [m/s²], \( \vec{F} \): resultierende Kraft [N]
}

\subsection{Das zweite Newtonsche Axiom}

\begin{displayformula}
	(lex secunda oder Aktionsprinzip). Die zeitliche Änderung des Impulses ist gleich der  
	resultierenden Kraft, die auf einen Körper wirkt.
\end{displayformula}

\begin{displayformula}
	Impuls
	\[
	\vec{p} = m \cdot \vec{v}
	\]
\end{displayformula}
\formulalegend{
	\( \vec{p} \): Impuls [kg·m/s], \( m \): Masse [kg], \( \vec{v} \): Geschwindigkeit [m/s]
}

\begin{displayformula}
	\[
	\sum_{i} \vec{F}_i = m \cdot \vec{a}
	\]
\end{displayformula}
\formulalegend{
	\( \sum_{i} \vec{F}_i \): Summe der Kräfte auf einen Körper [N], \( m \): Masse [kg], \( \vec{a} \): Beschleunigung [m/s²]
}

\begin{displayformula}
	\[
	m = \frac{m_{\text{Ruhe}}}{\sqrt{1 - \frac{v^2}{c^2}}}
	\]
\end{displayformula}
\formulalegend{
	\( m \): relativistische Masse [kg], \( m_{\text{Ruhe}} \): Ruhemasse [kg], \( v \): Geschwindigkeit [m/s], \( c \): Lichtgeschwindigkeit [m/s]
}

\begin{displayformula}
	Gravitationskraft
	\[
	F_G = G \cdot \frac{m_1 \cdot m_2}{r^2}
	\]
\end{displayformula}
\formulalegend{
	\( F_G \): Gravitationskraft [N], \( G \): Gravitationskonstante [m³/kg·s²], \( m_1, m_2 \): Massen der Körper [kg], \( r \): Abstand [m]
}

\subsection{Das dritte Newtonsche Axiom}

\begin{displayformula}
	\[
	\vec{F}_{12} = -\vec{F}_{21}
	\]
\end{displayformula}
\formulalegend{
	\( \vec{F}_{12} \): Kraft von Körper 1 auf 2 [N], \( \vec{F}_{21} \): Gegenkraft von 2 auf 1 [N]
}
\newpage


\section{Kontaktkräfte und weitere Arten von Kräften}

\begin{displayformula}
	Gewichtskraft \( F_G \)
	\[
	F_G = m \cdot g
	\]
\end{displayformula}
\formulalegend{
	\( F_G \): Gewichtskraft [N], \( m \): Masse [kg], \( g \): Erdbeschleunigung [m/s²]
}

\begin{displayformula}
	Normalkraft \( F_N \) (Immer senkrecht zum Untergrund)
	\[
	F_N = F_G
	\]
\end{displayformula}
\formulalegend{
	\( F_N \): Normalkraft [N], \( F_G \): Gewichtskraft [N]
}

\begin{displayformula}
	Reibungskraft \( F_R \)
	\[
	F_R = \mu \cdot F_N
	\]
\end{displayformula}
\formulalegend{
	\( F_R \): Reibungskraft [N], \( \mu \): Reibungskoeffizient, \( F_N \): Normalkraft [N]
}

\begin{displayformula}
	Hangabtriebskraft \( F_H \)
	\[
	F_H = m \cdot g \cdot \sin\alpha
	\]
	\( F_N \) wird kleiner
	\[
	F_N = m \cdot g \cdot \cos\alpha
	\]
\end{displayformula}
\formulalegend{
	\( F_H \): Hangabtriebskraft [N], \( F_N \): Normalkraft [N], \( m \): Masse [kg], \( g \): Erdbeschleunigung [m/s²], \( \alpha \): Neigungswinkel
}

\begin{displayformula}
	Federkraft
	\[
	F_{\text{Zug}} = K_F \cdot x
	\]
	\[
	F_{\text{Feder}} = -K_F \cdot x
	\]
\end{displayformula}
\formulalegend{
	\( F_{\text{Zug}} \): Zugkraft an der Feder [N], \( F_{\text{Feder}} \): Rückstellkraft der Feder [N], \( K_F \): Federkonstante [N/m], \( x \): Auslenkung [m]
}

\begin{displayformula}
	Zentripetalkraft \( \vec{F}_{ZP} \)
	\[
	\vec{F}_{ZP} = -m \cdot \omega^2 \cdot \vec{r} = \frac{m \cdot v^2}{r}
	\]
\end{displayformula}
\formulalegend{
	\( \vec{F}_{ZP} \): Zentripetalkraft [N], \( m \): Masse [kg], \( \omega \): Winkelgeschwindigkeit [rad/s], \( \vec{r} \): Radiusvektor [m], \( v \): Bahngeschwindigkeit [m/s]
}

\begin{displayformula}
	Luftwiderstandskraft
	\[
	F_W = \frac{1}{2} c_W \cdot \rho \cdot A \cdot v^2
	\]
	Vereinfacht
	\[
	F_W = b \cdot v^2
	\]
\end{displayformula}
\formulalegend{
	\( F_W \): Luftwiderstand [N], \( c_W \): Widerstandsbeiwert, \( \rho \): Dichte der Luft [kg/m³], \( A \): Querschnittsfläche [m²], \( v \): Geschwindigkeit [m/s], \( b \): Reibungskoeffizient [kg/m]
}
\newpage

\subsection{Trägheits- und Scheinkräfte}

\begin{displayformula}
	Trägheitskraft
	\[
	\vec{F}_T = -m \cdot \vec{a}_B
	\]
\end{displayformula}
\formulalegend{
	\( \vec{F}_T \): Trägheitskraft [N], \( m \): Masse [kg], \( \vec{a}_B \): Beschleunigung des Bezugssystems [m/s²]
}

\begin{displayformula}
	Zentrifugalkraft
	\[
	\vec{F}_{ZF} = m \cdot \omega^2 \cdot \vec{r} = - \vec{F}_{ZP}
	\]
\end{displayformula}
\formulalegend{
	\( \vec{F}_{ZF} \): Zentrifugalkraft [N], \( m \): Masse [kg], \( \omega \): Winkelgeschwindigkeit [rad/s], \( \vec{r} \): Radiusvektor [m]
}

\begin{displayformula}
	Corioliskraft
	\[
	\vec{F}_{\mathrm{Cor}} = 2m\, \vec{v} \times \vec{\omega} = 2m\, \lVert \vec{v} \rVert \lVert \vec{\omega} \rVert \sin(\vec{v}; \vec{\omega})
	\]
	\[
	\vec{a}_{\mathrm{Cor}} = 2 \cdot \vec{v}_0 \times \vec{\omega}
	\]
\end{displayformula}
\formulalegend{
	\( \vec{F}_{\mathrm{Cor}} \): Corioliskraft [N], \( m \): Masse [kg], \( \vec{v} \): Geschwindigkeit [m/s], \( \vec{\omega} \): \\ Winkelgeschwindigkeit [rad/s]
}
\newpage

\section{Der Massenmittelpunkt}

\begin{displayformula}
	Drehmoment \( \vec{M} \)
	\[
	\vec{M} = r \cdot \vec{F}
	\]
\end{displayformula}
\formulalegend{
	\( \vec{M} \): Drehmoment [Nm], \( r \): Hebelarm [m], \( \vec{F} \): angreifende Kraft [N]
}

\begin{displayformula}
	Statisches Problem
	\[
	\sum F_i = 0
	\]
	\[
	\sum M_i = 0
	\]
\end{displayformula}
\formulalegend{
	\( \sum F_i \): Summe aller Kräfte [N], \( \sum M_i \): Summe aller Momente [Nm]
}

\begin{displayformula}
	Massenmittelpunkt \( X_s \) bei 2 Teilchen
	\[
	X_s = \frac{X_1 \cdot m_1 + X_2 \cdot m_2}{m_1 + m_2}
	\]
\end{displayformula}
\formulalegend{
	\( X_s \): Schwerpunkt [m], \( X_1, X_2 \): Positionen der Massen [m], \( m_1, m_2 \): Massen [kg]
}

\begin{displayformula}
	Für \( n \) Teilchen gilt
	\[
	X_S = \frac{1}{m_{\text{ges}}} \sum m_i \cdot \vec{r}_i
	\]
\end{displayformula}
\formulalegend{
	\( X_S \): Massenmittelpunkt [m], \( m_i \): Masse des Teilchens [kg], \( \vec{r}_i \): Ort des Teilchens [m], \( m_{\text{ges}} \): Gesamtmasse [kg]
}

\begin{displayformula}
	Für \( \infty \) Teilchen gilt
	\[
	X_S = \frac{1}{m_{\text{ges}}} \int \vec{r} \, dm
	\]
\end{displayformula}
\formulalegend{
	\( X_S \): Massenmittelpunkt [m], \( \vec{r} \): Ortselement [m], \( m_{\text{ges}} \): Gesamtmasse [kg]
}
\newpage


\section{Arbeit und kinetische Energie}

\begin{displayformula}
	\[
	W = \vec{F} \cdot \vec{s} = |\vec{F}| \cdot |\vec{s}| \cdot \cos(\vec{F}; \vec{s})
	\]
\end{displayformula}
\formulalegend{
	\( W \): Arbeit [J], \( \vec{F} \): Kraft [N], \( \vec{s} \): Weg [m], \( \cos(\vec{F}; \vec{s}) \): Winkel zwischen Kraft und Weg
}

\begin{displayformula}
	Reibungsarbeit \( W_r \)
	\[
	W_r = F_r \cdot \Delta x = \mu F_N \cdot \Delta x
	\]
\end{displayformula}
\formulalegend{
	\( W_r \): Reibungsarbeit [J], \( F_r \): Reibungskraft [N], \( \mu \): Reibungskoeffizient, \( F_N \): Normalkraft [N], \( \Delta x \): Weg [m]
}

\begin{displayformula}
	Hubarbeit \( W_H \)
	\[
	W_H = m \cdot g \cdot h
	\]
	\[
	E_{\text{Pot}} = W_{\text{Pot}} = mgh
	\]
\end{displayformula}
\formulalegend{
	\( W_H \): Hubarbeit [J], \( E_{\text{Pot}} \): Potentielle Energie [J], \( m \): Masse [kg], \( g \): Erdbeschleunigung [m/s²], \( h \): Höhe [m]
}

\begin{displayformula}
	Beschleunigungsarbeit \( W_B \)
	\[
	W_B = F_B \cdot \Delta x = m \cdot a \cdot \Delta x = \frac{1}{2} mv^2
	\]
\end{displayformula}
\formulalegend{
	\( W_B \): Beschleunigungsarbeit [J], \( F_B \): Beschleunigende Kraft [N], \( \Delta x \): Weg [m], \( m \): Masse [kg], \( a \): Beschleunigung [m/s²], \( v \): Geschwindigkeit [m/s]
}

\begin{displayformula}
	Gesamtenergie bei geschlossenen Wegen (konservative Kräfte)
	\[
	W_{\text{ges}} = 0 = \oint \vec{F} \cdot d\vec{s}
	\]
\end{displayformula}
\formulalegend{
	\( W_{\text{ges}} \): Gesamtarbeit über geschlossene Bahn [J], \( \vec{F} \): Kraft [N], \( d\vec{s} \): Wegdifferenzial [m]
}
\newpage
\section{Verrichtete Arbeit bei geradliniger Bewegung mit ortsabhängiger Kraft}

\begin{displayformula}
	Einzelne Teilmengen
	\[
	dW_i = F_i \cdot ds_i
	\]
\end{displayformula}
\formulalegend{
	\( dW_i \): Infinitesimale Arbeit [J], \( F_i \): Kraft entlang des Wegs [N], \( ds_i \): Wegdifferenzial [m]
}

\begin{displayformula}
	Gesamte Arbeit zwischen \( S_1 \) und \( S_2 \)
	\[
	W = \int_{S_1}^{S_2} F(s) \, ds
	\]
\end{displayformula}
\formulalegend{
	\( W \): Arbeit [J], \( F(s) \): ortsabhängige Kraft [N], \( s \): Weg [m]
}
\newpage

\section{Leistung}

\begin{displayformula}
	Die Energieänderung eines Körpers pro Zeiteinheit heißt Leistung \( P \)
	\[
	P = \frac{\text{Verrichtete Arbeit}}{\text{Zeit}} = \vec{F} \cdot \vec{v}
	\]
	\[
	dW = \vec{F} \cdot d \vec{s} = \vec{F} \cdot \vec{v} \cdot dt
	\]
	\[
	\frac{dW}{dt} = \vec{F} \cdot \vec{v} = P
	\]
\end{displayformula}
\formulalegend{
	\( P \): Leistung [W], \( \vec{F} \): Kraft [N], \( \vec{v} \): Geschwindigkeit [m/s], \( t \): Zeit [s]
}
\newpage
\section{Energieerhaltung}

\begin{displayformula}
	Allgemeiner Energieerhaltungssatz: 
	\[
	E_{\text{ges}} = E_{\text{mech}} + E_{\text{wärme}} + E_{\text{chem}} + E_{\text{andere}} = \text{konstant}
	\]
\end{displayformula}

\begin{displayformula}
	Mechanischer Energieerhaltungssatz:
	\[
	E_{\text{ges}} = \sum_{i=1}^{n} E_{\text{Pot}, i} + \sum_{i=1}^{n} E_{\text{Kin}, i} = \text{konstant}, \quad \text{wenn } \vec{F}_{\text{ext}} = 0
	\]
	\[
	E_{\text{Pot}} = m \cdot g \cdot h
	\]
	\[
	E_{\text{Kin}} = \frac{1}{2} \cdot m \cdot v^2
	\]
\end{displayformula}
\formulalegend{
	\( E_{\text{ges}} \): Gesamte mechanische Energie [J], \( E_{\text{Pot}} \): Potentielle Energie [J], \( E_{\text{Kin}} \): Kinetische Energie [J], \( m \): Masse [kg], \( g \): Erdbeschleunigung [m/s²], \( h \): Höhe [m], \( v \): Geschwindigkeit [m/s]
}
\newpage
\section{Impuls und Impulserhaltung}

\begin{displayformula}
	Der Impuls \( \vec{p} \) einer Masse ist definiert als das Produkt aus der Masse \( m \) und ihrer Geschwindigkeit \( \vec{v} \):
	\[
	\vec{p} = m \cdot \vec{v}
	\]
\end{displayformula}
\formulalegend{
	\( \vec{p} \): Impuls [kg·m/s], \( m \): Masse [kg], \( \vec{v} \): Geschwindigkeit [m/s]
}

\begin{displayformula}
	Impulserhaltungssatz:
	\[
	\vec{p}_{\text{ges}} = \sum_i m_i \cdot \vec{v}_i = \text{konstant}, \quad \text{wenn } \vec{F}_{\text{ext}} = 0
	\]
\end{displayformula}
\formulalegend{
	\( \vec{p}_{\text{ges}} \): Gesamtimpuls [kg·m/s], \( m_i \): Massen [kg], \( \vec{v}_i \): Geschwindigkeiten [m/s], \( \vec{F}_{\text{ext}} \): äußere Kraft [N]
}
\newpage
\section{Stoßprozesse}

\begin{displayformula}
	Elastischer Stoß:
	\[
	\sum_i \frac{1}{2} m_i v_{i,\text{vor}}^2 = \sum_i \frac{1}{2} m_i v_{i,\text{nach}}^2
	\]
\end{displayformula}
\formulalegend{
	\( m_i \): Masse [kg], \( v_{i,\text{vor}} \): Geschwindigkeit vor dem Stoß [m/s], \( v_{i,\text{nach}} \): Geschwindigkeit nach dem Stoß [m/s]
}

\begin{displayformula}
	Inelastischer Stoß:
	\[
	\sum_i \frac{1}{2} m_i v_{i,\text{vor}}^2 = \sum_i \frac{1}{2} m_i v_{i,\text{nach}}^2 + \Delta W
	\]
\end{displayformula}
\formulalegend{
	\( \Delta W \): Energieverlust [J], Rest wie oben
}
\begin{displayformula}
	Vollständig inelastischer Stoß:
	\[
	\sum_i \frac{1}{2} m_i v_{i,\text{vor}}^2 = \Delta W
	\]
\end{displayformula}
\formulalegend{
	Alle kinetische Energie geht in andere Energieformen über (z.\,B. Wärme, Verformung)
}
\newpage
\subsubsection{Gerader, zentraler, elastischer Stoß, zweite Kugel in Ruhe}

\begin{displayformula}
	Impulserhaltung:
	\[
	m_1 \vec{v}_1 + m_2 \vec{v}_2 = m_1 \vec{v}_1' + m_2 \vec{v}_2'
	\]
\end{displayformula}
\formulalegend{
	\( m_1, m_2 \): Massen [kg], \( \vec{v}_1, \vec{v}_2 \): Geschwindigkeiten vor dem Stoß [m/s], \( \vec{v}_1', \vec{v}_2' \): Geschwindigkeiten danach [m/s]
}

\begin{displayformula}
	Energieerhaltung:
	\[
	\frac{1}{2} m_1 v_1^2 + \frac{1}{2} m_2 v_2^2 = \frac{1}{2} m_1 {v_1'}^2 + \frac{1}{2} m_2 {v_2'}^2
	\]
\end{displayformula}
\formulalegend{
	Kinetische Energie vor und nach dem Stoß ist gleich (elastischer Stoß)
}

\begin{displayformula}
	Geschwindigkeiten nach dem Stoß:
	\[
	v_1' = \frac{(m_1 - m_2) v_1 + 2 m_2 v_2}{m_1 + m_2}
	\]
	\[
	v_2' = \frac{(m_2 - m_1) v_2 + 2 m_1 v_1}{m_1 + m_2}
	\]
\end{displayformula}
\formulalegend{
	\( v_1', v_2' \): Endgeschwindigkeiten nach dem Stoß [m/s], \( m_1, m_2 \): Massen [kg], \( v_1, v_2 \): Anfangsgeschwindigkeiten [m/s]
}
\newpage
\section{Drehbewegungen}

\begin{displayformula}
	Die Länge eines Kreisbogens \( s \) ergibt sich aus dem Zusammenhang:
	\[
	s = r \cdot \varphi 
	\]
\end{displayformula}
\formulalegend{
	\( s \): Kreisbogenlänge [m], \( r \): Radius [m], \( \varphi \): Winkel im Bogenmaß [rad]
}

\begin{displayformula}
	Winkelgeschwindigkeit (Änderung des Drehwinkels pro Zeit)
	\[
	\omega = \frac{d\varphi(t)}{dt} \quad \text{mit } [\omega] = \text{rad/s}
	\]
\end{displayformula}
\formulalegend{
	\( \omega \): Winkelgeschwindigkeit [rad/s], \( \varphi(t) \): Winkel [rad], \( t \): Zeit [s]
}

\begin{displayformula}
	Winkelbeschleunigung (Änderung der Winkelgeschwindigkeit pro Zeit)
	\[
	\alpha = \frac{d^2 \varphi(t)}{dt^2} = \frac{d\omega(t)}{dt} \quad \text{mit } [\alpha] = \text{rad/s}^2
	\]
\end{displayformula}
\formulalegend{
	\( \alpha \): Winkelbeschleunigung [rad/s²], \( \omega(t) \): Winkelgeschwindigkeit [rad/s], \( \varphi(t) \): Winkel [rad]
}

\begin{displayformula}
	Bahngeschwindigkeit
	\[
	v_t = r \cdot \omega
	\]
\end{displayformula}
\formulalegend{
	\( v_t \): Bahngeschwindigkeit [m/s], \( r \): Radius [m], \( \omega \): Winkelgeschwindigkeit [rad/s]
}

\begin{displayformula}
	Tangentialbeschleunigung
	\[
	a_t = r \cdot \alpha
	\]
\end{displayformula}
\formulalegend{
	\( a_t \): Tangentialbeschleunigung [m/s²], \( r \): Radius [m], \( \alpha \): Winkelbeschleunigung [rad/s²]
}

\begin{displayformula}
	Zentripetalbeschleunigung
	\[
	a_n = -r \cdot \omega^2 = - \frac{v_t^2}{r}
	\]
\end{displayformula}
\formulalegend{
	\( a_n \): Zentripetalbeschleunigung [m/s²], \( r \): Radius [m], \( \omega \): Winkelgeschwindigkeit [rad/s], \( v_t \): Bahngeschwindigkeit [m/s]
}

\begin{displayformula}
	Grundformeln
	\[
	\varphi (t) = \varphi_0 + \omega t + \frac{1}{2} \alpha t^2
	\]
	\[
	\omega (t) = \omega_0 + \alpha t
	\]
	\[
	\alpha (t) = \alpha
	\]
\end{displayformula}
\formulalegend{
	\( \varphi(t) \): Winkel [rad], \( \varphi_0 \): Anfangswinkel [rad], \( \omega \): Anfangswinkelgeschwindigkeit [rad/s], \( \alpha \): konstante Winkelbeschleunigung [rad/s²], \( t \): Zeit [s]
}
\newpage
\begin{displayformula}
	\[
	\omega^2 = \omega_0^2 + 2\alpha \Delta\varphi
	\]
	\[
	\omega = 2\pi f = \frac{2\pi}{T}
	\]
	\[
	P = M \cdot \omega
	\]
\end{displayformula}
\formulalegend{
	\( \omega \): Winkelgeschwindigkeit [rad/s], \( \omega_0 \): Anfangswinkelgeschwindigkeit [rad/s], \( \alpha \): Winkelbeschleunigung [rad/s²], \( \Delta \varphi \): Winkeländerung [rad], \( f \): Frequenz [Hz], \( T \): Periodendauer [s], \( P \): Leistung [W], \( M \): Drehmoment [Nm]
}

\subsection{Die Kinetische Energie der Drehbewegung}

\begin{displayformula}
	\[
	E_{\text{kin}} = \sum_i \frac{1}{2} m_i r_i^2 \cdot \omega^2 = \frac{1}{2} I \omega^2
	\]
\end{displayformula}
\formulalegend{
	\( E_{\text{kin}} \): Rotationsenergie [J], \( m_i \): Masse [kg], \( r_i \): Abstand zur Drehachse [m], \( \omega \): Winkelgeschwindigkeit [rad/s], \( I \): Trägheitsmoment [kg·m²]
}
\newpage
\begin{figure}[h]
	\centering
	\includegraphics[width=\linewidth]{/Users/cematas/Documents/Physik/Formelsammlung 2/Images/Vergleich.pdf}
	\label{fig:beispiel}
\end{figure}
\newpage
\section{Massenträgheitsmomente}

\begin{displayformula}
	Massenträgheitsmoment
	\[
	I = \sum \frac{1}{2} m_i r_i^2 
	\]
	\[
	I_{\text{ges}} = \sum_i I_i
	\]
\end{displayformula}
\formulalegend{
	\( I \): Trägheitsmoment eines Körpers [kg·m²], \( m_i \): Masse [kg], \( r_i \): Abstand zur Drehachse [m]
}

\begin{displayformula}
	Kontinuierliche Masseverteilungen
	\[
	I = \int r^2 \, dm
	\]
\end{displayformula}
\formulalegend{
	\( I \): Trägheitsmoment [kg·m²], \( r \): Abstand zur Drehachse [m], \( dm \): infinitesimale Masse [kg]
}
\begin{displayformula}
	Massiver, homogener Zylinder (Masse \( m \); Radius \( r_a \))
	\[
	I = \frac{1}{2} m \cdot r_a^2
	\]
\end{displayformula}
\formulalegend{
	\( I \): Trägheitsmoment [kg·m²], \( m \): Masse [kg], \( r_a \): Außenradius [m]
}

\begin{displayformula}
	Hohlzylinder (Masse \( m \); Innenradius \( r_i \); Außenradius \( r_a \))
	\[
	I = \frac{1}{2} m \cdot (r_a^2 + r_i^2)
	\]
\end{displayformula}
\formulalegend{
	\( I \): Trägheitsmoment [kg·m²], \( m \): Masse [kg], \( r_a \): Außenradius [m], \( r_i \): Innenradius [m]
}

\begin{displayformula}
	Dünnwandiger, hohler Zylinder (Radius \( r_a \))
	\[
	I = m \cdot r_a^2
	\]
\end{displayformula}
\formulalegend{
	\( I \): Trägheitsmoment [kg·m²], \( m \): Masse [kg], \( r_a \): Radius [m]
}

\begin{displayformula}
	Dünner Stab (Länge \( l \); durch die Mitte gedreht)
	\[
	I = \frac{1}{12} m \cdot l^2
	\]
\end{displayformula}
\formulalegend{
	\( I \): Trägheitsmoment [kg·m²], \( m \): Masse [kg], \( l \): Länge [m]
}

\begin{displayformula}
	Dünner Stab (Drehachse durch das Ende)
	\[
	I = \frac{1}{3} m \cdot l^2
	\]
\end{displayformula}
\formulalegend{
	\( I \): Trägheitsmoment [kg·m²], \( m \): Masse [kg], \( l \): Länge [m]
}

\begin{displayformula}
	Bei versetzter Drehachse
	\[
	E_{\text{kin}} = \frac{1}{2} I_s \omega^2 + \frac{1}{2} m r^2 \omega^2 = \frac{1}{2} (I_s + m r^2) \omega^2
	\]
	Steiner
	\[
	I_p = I_s + m r^2
	\]
\end{displayformula}
\formulalegend{
	\( E_{\text{kin}} \): Kinetische Energie der Rotation [J], \( I_s \): Trägheitsmoment um Schwerpunktachse [kg·m²], \( I_p \): Trägheitsmoment um Parallelachse [kg·m²], \( m \): Masse [kg], \( r \): Abstand der Achsen [m], \( \omega \): Winkelgeschwindigkeit [rad/s]
}
\newpage
\section{Das zweite Newtonsche Axiom für Drehbewegungen}

\begin{displayformula}
	\[
	\vec{M} = I \cdot \vec{\alpha}
	\]
\end{displayformula}
\formulalegend{
	\( \vec{M} \): Drehmoment [Nm], \( I \): Trägheitsmoment [kg·m²], \( \vec{\alpha} \): Winkelbeschleunigung [rad/s²]
}

\begin{displayformula}
	Drehmoment über Kreuzprodukt
	\[
	\vec{M} = \vec{r} \times \vec{F}
	\]
\end{displayformula}
\formulalegend{
	\( \vec{M} \): Drehmoment [Nm], \( \vec{r} \): Hebelarm [m], \( \vec{F} \): Kraft [N]
}

\begin{displayformula}
	Tangentialbeschleunigung
	\[
	a_t = \vec{\alpha} \cdot \vec{r} \quad \Rightarrow \quad \vec{F} = m \cdot \vec{\alpha} \times \vec{r}
	\]
\end{displayformula}
\formulalegend{
	\( a_t \): Tangentialbeschleunigung [m/s²], \( \vec{\alpha} \): Winkelbeschleunigung [rad/s²], \( \vec{r} \): Radiusvektor [m], \( \vec{F} \): Kraft [N], \( m \): Masse [kg]
}
\newpage
\subsection{Statisches Gleichgewicht}

\begin{displayformula}
	\[
	\vec{F} = m \cdot \vec{a} = 0
	\]
	\[
	\vec{M} = I \cdot \vec{\alpha} = 0
	\]
\end{displayformula}
\formulalegend{
	Statische Bedingungen: keine Beschleunigung, keine Winkelbeschleunigung. Kräfte- und Momentengleichgewicht.\\
	\( \vec{F} \): resultierende Kraft [N], \( \vec{a} \): Beschleunigung [m/s²], \( \vec{M} \): Drehmoment [Nm], \( \vec{\alpha} \): Winkelbeschleunigung [rad/s²]
}
\newpage
\subsection{Die kinetische Energie rollender Körper}

\begin{displayformula}
	\[
	E_{\text{kin}} = \frac{1}{2} I_S \omega^2 + \frac{1}{2} m v_S^2
	\]
\end{displayformula}
\formulalegend{
	\( E_{\text{kin}} \): Gesamtenergie [J], \( I_S \): Trägheitsmoment um Schwerpunkt [kg·m²], \( \omega \): Winkelgeschwindigkeit [rad/s], \( m \): Masse [kg], \( v_S \): Schwerpunktsgeschwindigkeit [m/s]
}

\begin{displayformula}
	Vollzylinder auf schiefer Ebene (Neigung \( \beta \)):
	\[
	E_{\text{pot}} = E_{\text{kin}}
	\]
	Geschwindigkeit nach Strecke \( x \):
	\[
	v_x^2 = \frac{4}{3} g \cdot x \cdot \sin\beta
	\]
	Beschleunigung:
	\[
	a = \frac{2}{3} g \cdot \sin\beta
	\]
\end{displayformula}
\formulalegend{
	\( E_{\text{pot}} \): Potentielle Energie [J], \( v_x \): Geschwindigkeit [m/s], \( g \): Erdbeschleunigung [m/s²], \( x \): zurückgelegte Strecke [m], \( \beta \): Neigungswinkel [rad], \( a \): Beschleunigung [m/s²]
}
\newpage
\section{Drehimpuls und Drehimpulserhaltung}

\begin{displayformula}
	Drehimpuls
	\[
	L = I \cdot \omega = \vec{r} \times \vec{p}
	\]
	\[
	\vec{L}_{\text{ges}} = \vec{L}_{\text{Bahn}} + \vec{L}_{\text{Spin}} = m \cdot \vec{r}_S \times \vec{v}_S + \vec{L}_{\text{Spin}}
	\]
\end{displayformula}
\formulalegend{
	\( L \): Drehimpuls [kg·m²/s], \( I \): Trägheitsmoment [kg·m²], \( \omega \): Winkelgeschwindigkeit [rad/s], \( \vec{r} \): Ort [m], \( \vec{p} \): Impuls [kg·m/s], \( \vec{v}_S \): Geschwindigkeit Schwerpunkt [m/s]
}

\begin{displayformula}
	Drehimpulserhaltung
	\[
	\vec{L}_{\text{ges}} = \sum_i I_i \cdot \omega_i = \text{konstant}, \quad \text{wenn } \vec{M}_{\text{ges}} = 0
	\]
\end{displayformula}
\formulalegend{
	\( \vec{L}_{\text{ges}} \): Gesamtdrehimpuls [kg·m²/s], \( \vec{M}_{\text{ges}} \): Summe der äußeren Drehmomente [Nm]
}
\newpage
\section{Schwingungen}

\subsection{Ungedämpfte, freie und harmonische Schwingungen}

\begin{displayformula}
	Auslenkung, Geschwindigkeit, Beschleunigung
	\[
	y(t) = A \cdot \cos(\omega_0 t + \delta)
	\]
	\[
	v(t) = -\omega_0 A \cdot \sin(\omega_0 t + \delta)
	\]
	\[
	a(t) = -\omega_0^2 A \cdot \cos(\omega_0 t + \delta)
	\]
\end{displayformula}
\formulalegend{
	\( y(t) \): Auslenkung [m], \( v(t) \): Geschwindigkeit [m/s], \( a(t) \): Beschleunigung [m/s²], \( A \): Amplitude [m], \( \omega_0 \): Kreisfrequenz [rad/s], \( \delta \): Phasenverschiebung [rad], \( t \): Zeit [s]
}

\begin{displayformula}
	Kreisfrequenz
	\[
	\omega_0 = 2\pi f_0 = \frac{2\pi}{T_0} = \sqrt{\frac{k_F}{m}}
	\]
\end{displayformula}
\formulalegend{
	\( \omega_0 \): Kreisfrequenz [rad/s], \( f_0 \): Frequenz [Hz], \( T_0 \): Periodendauer [s], \( k_F \): Federkonstante [N/m], \( m \): Masse [kg]
}

\begin{displayformula}
	Energie des harmonischen Oszillators
	\[
	E_{\text{mech}} = \frac{1}{2} k_F \cdot A^2
	\]
\end{displayformula}
\formulalegend{
	\( E_{\text{mech}} \): Mechanische Energie [J], \( k_F \): Federkonstante [N/m], \( A \): Amplitude [m]
}

\begin{displayformula}
	Vertikaler Federschwinger
	\[
	\omega_0 = \sqrt{\frac{k_F}{m}}
	\]
\end{displayformula}
\formulalegend{
	\( \omega_0 \): Kreisfrequenz [rad/s], \( k_F \): Federkonstante [N/m], \( m \): Masse [kg]
}

\begin{displayformula}
	Mathematisches Pendel
	\[
	\ddot{\theta}(t) + \frac{g}{l} \sin\theta(t) = 0
	\]
	Linearisiert:
	\[
	\ddot{\theta}(t) + \frac{g}{l} \theta(t) = 0
	\]
	\[
	\omega_0 = \sqrt{\frac{g}{l}}
	\]
\end{displayformula}
\formulalegend{
	\( \theta(t) \): Winkel [rad], \( g \): Erdbeschleunigung [m/s²], \( l \): Pendellänge [m], \( \omega_0 \): Kreisfrequenz [rad/s]
}

\begin{displayformula}
	Drehpendel / Torsionspendel
	\[
	\ddot{\theta}(t) + \frac{\kappa}{I} \theta(t) = 0
	\quad \Rightarrow \quad \omega_0 = \sqrt{\frac{\kappa}{I}}
	\]
\end{displayformula}
\formulalegend{
	\( \theta(t) \): Auslenkwinkel [rad], \( \kappa \): Drehfederkonstante [Nm], \( I \): Trägheitsmoment [kg·m²]
}

\begin{displayformula}
	Physikalisches Pendel
	\[
	\ddot{\theta}(t) + \frac{s m g}{I_p} \sin\theta(t) = 0
	\quad \Rightarrow \quad \omega_0 = \sqrt{\frac{s m g}{I_p}}
	\]
\end{displayformula}
\formulalegend{
	\( \theta(t) \): Winkel [rad], \( s \): Abstand zur Drehachse [m], \( m \): Masse [kg], \( g \): Erdbeschleunigung [m/s²], \( I_p \): Trägheitsmoment bezogen auf Drehachse [kg·m²]
}

\begin{displayformula}
	Elastischer Schwingkreis
	\[
	\ddot{Q}(t) + \frac{1}{LC} Q(t) = 0
	\quad \Rightarrow \quad \omega_0 = \sqrt{\frac{1}{LC}}
	\]
\end{displayformula}
\formulalegend{
	\( Q(t) \): Ladung [C], \( L \): Induktivität [H], \( C \): Kapazität [F], \( \omega_0 \): Kreisfrequenz [rad/s]
}
\newpage

\subsection{Gedämpfte Schwingungen}

\begin{displayformula}
	DGL für Feder-Masse-Dämpfungssystem:
	\[
	\ddot{y}(t) + 2\delta \dot{y}(t) + \omega_0^2 y(t) = 0
	\quad \text{mit } 2\delta = \frac{b}{m}
	\]
\end{displayformula}
\formulalegend{
	\( y(t) \): Auslenkung [m], \( \delta \): Abklingkonstante [1/s], \( b \): Dämpfungskonstante [kg/s], \( m \): Masse [kg], \( \omega_0 \): ungedämpfte Kreisfrequenz [rad/s]
}

\begin{displayformula}
	Dämpfungsgrad
	\[
	D = \frac{\delta}{\omega_0}
	\]
\end{displayformula}
\formulalegend{
	\( D \): Dämpfungsgrad, \( \delta \): Abklingkonstante [1/s], \( \omega_0 \): Kreisfrequenz [rad/s]
}

\begin{displayformula}
	Lösung der charakteristischen Gleichung
	\[
	\lambda_{1,2} = -\delta \pm \sqrt{\delta^2 - \omega_0^2}
	\]
\end{displayformula}
\formulalegend{
	\( \lambda \): Eigenwerte, \( \delta \): Abklingkonstante [1/s], \( \omega_0 \): Kreisfrequenz [rad/s]
}
\newpage
\begin{figure}[h]
	\centering
	\includegraphics[width=\linewidth]{/Users/cematas/Documents/Physik/Formelsammlung 2/Images/DGL.pdf}
\end{figure}
\newpage
\subsection{Energie des gedämpften Oszillators}

\begin{displayformula}
	Schwach gedämpft (Näherung \( \omega_d \approx \omega_0 \)):
	\[
	E_{\text{mech}} = \frac{1}{2} m \cdot \omega_0^2 \cdot A^2
	\]
\end{displayformula}
\formulalegend{
	\( E_{\text{mech}} \): Energie [J], \( m \): Masse [kg], \( \omega_0 \): Kreisfrequenz [rad/s], \( A \): Amplitude [m]
}

\begin{displayformula}
	Stärker gedämpft:
	\[
	E_{\text{mech}} = \frac{1}{2} m \cdot \omega_d^2 \cdot A^2
	\]
\end{displayformula}
\formulalegend{
	\( \omega_d \): gedämpfte Eigenfrequenz [rad/s]
}

\subsection{Güte}

\begin{displayformula}
	Gütefaktor
	\[
	Q = \frac{1}{2D} = \omega_0 \cdot \frac{m}{b}
	\]
\end{displayformula}
\formulalegend{
	\( Q \): Gütefaktor, \( D \): Dämpfungsgrad, \( \omega_0 \): Kreisfrequenz [rad/s], \( m \): Masse [kg], \( b \): Dämpfungskonstante [kg/s]
}
\newpage
\section{Wellen}

\begin{displayformula}
	Eindimensionale Wellengleichung
	\[
	\frac{\partial^2 y(z,t)}{\partial t^2} = \frac{F_s}{A \rho} \cdot \frac{\partial^2 y(z,t)}{\partial z^2}
	\]
\end{displayformula}
\formulalegend{
	\( y(z,t) \): Auslenkung [m], \( F_s \): Zugkraft [N], \( A \): Querschnittsfläche [m²], \( \rho \): Dichte [kg/m³]
}

\begin{displayformula}
	Harmonische Wellenfunktion
	\[
	y(z, t) = A \cdot \cos(\omega t - kz + \delta)
	\]
	\[
	k = \frac{2\pi}{\lambda}, \quad c = \frac{\lambda}{T} = \frac{\omega}{k} = \nu \cdot \lambda
	\]
\end{displayformula}
\formulalegend{
	\( y(z,t) \): Auslenkung [m], \( A \): Amplitude [m], \( \omega \): Kreisfrequenz [rad/s], \( k \): Wellenzahl [rad/m], \( \delta \): Phase [rad], \( \lambda \): Wellenlänge [m], \( T \): Periodendauer [s], \( \nu \): Frequenz [Hz], \( c \): Ausbreitungsgeschwindigkeit [m/s]
}

\end{document}
